\documentclass[11pt,a4paper]{article}
\usepackage[utf8]{inputenc}
\usepackage[T1]{fontenc}
\usepackage[portuguese]{babel}
\usepackage{geometry}
\usepackage{fontawesome5}
\usepackage{xcolor}
\usepackage{hyperref}
\usepackage{titlesec}
\usepackage{enumitem}
\usepackage{setspace}
\usepackage{fancyhdr}
\usepackage{lastpage}
\usepackage{tikz}
\usepackage{graphicx}

% Carrega as configurações personalizadas
\input{config}

% Configurações de hyperref
\hypersetup{
    colorlinks=true,
    linkcolor=primary,
    urlcolor=primary,
    citecolor=primary
}

% Configurações de títulos
\titleformat{\section}
    {\Large\bfseries\color{primary}}
    {}{0em}{}[\titlerule]

\titleformat{\subsection}
    {\large\bfseries\color{secondary}}
    {}{0em}{}

% Configurações de listas
\setlist[itemize]{leftmargin=*,itemsep=0.1em,topsep=0.1em}
\setlist[enumerate]{leftmargin=*,itemsep=0.1em,topsep=0.1em}

% Espaçamento
\onehalfspacing

% Cabeçalho
\pagestyle{fancy}
\fancyhf{}
\fancyfoot[C]{\thepage\ / \pageref{LastPage}}
\renewcommand{\headrulewidth}{0pt}

\begin{document}

% Cabeçalho principal com design moderno e fundo escuro
\begin{tikzpicture}[remember picture,overlay]
    \fill[black] (-\hoffset-\oddsidemargin-2in,-4cm) rectangle (\paperwidth+\hoffset+\oddsidemargin+2in,3cm);
\end{tikzpicture}

\vspace{-2cm}
\begin{center}
    \begin{minipage}[t]{0.1\textwidth}
        % Coluna da esquerda - Foto
        \begin{center}
            \begin{tikzpicture}
                \clip (0,0) circle (1.5cm);
                \node at (0,0) {\includegraphics[width=3cm]{Foto.png}};
            \end{tikzpicture}
        \end{center}
    \end{minipage}
    \hfill
    \begin{minipage}[t]{0.7\textwidth}
        % Coluna da direita - Dados pessoais
        \begin{flushleft}
            {\Huge\bfseries\color{white} \nome}\\[0.4em]
            {\large\color{white} \titulo}\\[1.2em]
            
            % Dados pessoais em duas colunas
            \begin{minipage}[t]{0.48\textwidth}
                \begin{tabular}{ll}
                    \color{white}\faIcon{envelope} & \color{white}\href{mailto:\email}{\email} \\
                    \color{white}\faIcon{phone} & \color{white}\telefone
                \end{tabular}
            \end{minipage}
            \hfill
            \begin{minipage}[t]{0.48\textwidth}
                \begin{tabular}{ll}
                    \color{white}\faIcon{linkedin} & \color{white}\href{https://www.\linkedin}{\linkedin} \\
                    \color{white}\faIcon{github} & \color{white}\href{https://\github}{\github}
                \end{tabular}
            \end{minipage}
            
            \vspace{0.5em}
            % Location below
            \begin{center}
                \color{white}\faIcon{map-marker-alt} \color{white} San Sebastián, Basque Country, Spain \\
                \color{white} Available to relocate to the United Kingdom
            \end{center}
        \end{flushleft}
    \end{minipage}
\end{center}

\vspace{2.5em}

\vspace{1em}

% Education
\section{Education}

\subsection{Master's Degree in Automation, Electronics and Industrial Control}
\textit{University of Deusto} \hfill \textit{2019 - 2020}
\begin{itemize}
    \item Master's Degree in Automation, Electronics and Industrial Control
    \item Specialized in industrial automation, embedded systems, and robotics
    \item \textit{GPA: 7.6/10.0 (3.8/5.0 international scale)}
\end{itemize}

\subsection{Bachelor's Degree in Electrical and Electronics Engineering}
\textit{Universidade Federal de Pernambuco} \hfill \textit{2013 - 2018}
\begin{itemize}
    \item Bachelor's Degree in Electrical and Electronics Engineering
    \item Focus on embedded systems, FPGA design, and biomedical electronics
\end{itemize}

% Professional Summary
\section{Professional Summary}
\resumo

% Career Objective
\section{Career Objective}
\begin{itemize}
    \item Seeking opportunities in FPGA design, embedded systems development, and industrial automation
    \item Passionate about real-time systems, computer vision, and robotics integration
    \item Interested in leading-edge technologies including edge computing and AI acceleration
\end{itemize}

% Professional Experience
\section{Professional Experience}

\subsection{Electronics Department Lead}
\textit{INZU Group (Ikergune)} \hfill \textit{January 2023 - Present}
\textit{San Sebastián, Basque Country, Spain}
\begin{itemize}
    \item Led development of advanced vision systems utilizing NIR and SWIR spectral cameras for industrial process analysis, implementing custom C\# software for real-time image acquisition and processing with RTSP and HTTP streaming protocol support
    \item Designed and implemented FPGA IP cores for convolutional kernel acceleration on Xilinx UltraScale+ devices, creating high-throughput data pipelines for real-time image processing applications
    \item Integrated FPGA acceleration into embedded edge computing platforms running Yocto Linux, developing C++ drivers and middleware for FPGA interface and data flow management
    \item Engineered industrial communication solutions integrating IO-Link and HTTP protocols with machine networks, implementing robust protocol stacks for reliable data exchange and control
    \item Developed custom sensor systems using Infineon microcontrollers with FreeRTOS, incorporating PID control loops and SPI peripheral communication, with PCB design via Altium Designer, for particulate matter measurement applications
    \item Designed and implemented stepper motor control systems using XMC microcontrollers integrated with FreeRTOS, developing trajectory control algorithms with acceleration and deceleration profiles in C/C++ for precise motion management
\end{itemize}
\textit{Skills:} Yocto Project, Verilog, FPGA, C++, Embedded C, HDL Designer, FPGA prototyping, Bash, VHDL, Field-Programmable Gate Arrays, CMake, Bare-Metal Embedded, Device Drivers, Data Acquisition, Embedded Linux, Embedded Systems, Hardware Description Language, Firmware, I2C, IO-Link, Kernel, Devicetree, U-Boot, Bootloader, OpenCV, Computer Vision, Machine Learning

\subsection{Software Engineer}
\textit{INZU Group (Ikergune)} \hfill \textit{August 2022 - January 2023}
\textit{San Sebastián, Basque Country, Spain}
\begin{itemize}
    \item Designed and implemented comprehensive integration pipeline between custom CAD/CAM system and robotic simulation API, utilizing C++ to automate simulation, path planning, and KUKA code generation
    \item Developed both API interface and post-processor components, managing complete workflow from trajectory acquisition to robot-compatible output generation
    \item Adapted toolpaths from upstream CAM software to KUKA syntax, incorporating kinematic validation, axis limits, and machine-specific parameters
    \item Validated generated programs through simulation and deployment in robotic cell environment, ensuring precise and collision-free execution
\end{itemize}
\textit{Skills:} C++, CMake, Application Programming Interfaces (API), Automation, Industrial Robots, Robot Framework, CAD/CAM Software

\subsection{Robotics Integration Engineer}
\textit{INZU Group (Ikergune)} \hfill \textit{May 2021 - August 2022}
\textit{San Sebastián, Basque Country, Spain}
\begin{itemize}
    \item Developed and configured complete KUKA robot control system using WorkVisual, including programming of robot motion sequences and comprehensive system diagnostics
    \item Engineered safety systems for robotic cell, implementing both robot-integrated safety functions and safety zone monitoring using PROFIsafe protocol stack over PROFINET, ensuring compliance with industrial safety standards (SIL/PL)
    \item Performed precision and repeatability verification using laser tracker systems, developing custom C\# scripts to automate measurement sequences, data acquisition, and error analysis
\end{itemize}
\textit{Skills:} Robot Framework, Protocol Stacks, Profinet, PROFIsafe, IO-Link, EtherCAT, C++, C\#

\subsection{Automation \& Software Developer}
\textit{INZU Group (Ikergune)} \hfill \textit{September 2020 - May 2021}
\textit{Elgóibar, Basque Country, Spain}
\begin{itemize}
    \item Developed Python scripts to automate and optimize internal industrial processes, enhancing efficiency and repeatability in manufacturing workflows
    \item Built C\# applications to integrate user interfaces and machine controls, streamlining human-machine interaction and data flow management
    \item Participated in computer vision projects, contributing to development of inspection and monitoring systems using cameras and image processing techniques
    \item Contributed to design and implementation of patented industrial control system for automated powder feeding equipment, integrating motor control, pressure regulation, and sensor feedback loops
    \item Developed control logic using PLCs and integrated SCADA systems for real-time process monitoring, data acquisition, and system supervision
\end{itemize}
\textit{Skills:} SCADA, PLC Siemens, PLC Programming, Programmable Logic Controller (PLC), SIMATIC STEP 7, PyTorch, Tkinter, Python, Anaconda, Windows Presentation Foundation (WPF), WPF Development, C\#

\subsection{System Engineer}
\textit{Aingura IIoT} \hfill \textit{February 2020 - September 2020}
\textit{Bilbao, Basque Country, Spain}
\begin{itemize}
    \item Developed embedded C code for Texas Instruments DSP to acquire digital sensor signals and communicate with C\# application via SCI (Serial Communication Interface)
    \item Built C\# application to acquire data from industrial KUKA robot over TCP/IP (via KUKAVARPROXY) and interface with DSP for synchronized data handling and storage
    \item Created LabVIEW RT modules on CompactRIO for real-time acquisition of digital signals and robot position data via TCP/IP, including data management and logging
    \item Integrated external sensors to capture physical signals related to robot movement, ensuring signal consistency across different acquisition systems
    \item Performed system calibration, trajectory setup, and signal comparison to validate synchronization and analyze timing behavior across hardware and software layers
\end{itemize}
\textit{Skills:} Embedded Systems, Signal Processing, Internet Protocol Suite (TCP/IP), C\#, LabVIEW, Kuka Robot, Data Analysis, MATLAB, NI LabVIEW, TI DSPs, Kuka Robots, Industrial Robots, Embedded C, Object-Oriented Programming (OOP)

\subsection{Embedded Systems \& Telemetry Engineer – Aerospace Subsystems}
\textit{Asa Branca Rocket Design} \hfill \textit{March 2018 - July 2019}
\textit{Recife, Pernambuco, Brazil}
\begin{itemize}
    \item Developed telemetry subsystem for rockets, drones, and cubesats, focusing on real-time sensor data acquisition, processing, and inter-subsystem communication
    \item Programmed firmware in C using ChibiOS/RT on STM32 boards, ensuring scalable and deterministic embedded architecture for critical flight operations
    \item Integrated sensors via SPI, I²C, and UART, optimizing driver-level communication and RTOS task management
\end{itemize}
\textit{Skills:} Embedded Systems, Linux, Real-Time Operating Systems (RTOS), C, Task Management, Data Acquisition, Satellite Systems Engineering, ChibiOS, Universal Asynchronous Receiver/Transmitter (UART), STM32, SPI, I2C, Spacecraft Telemetry

\subsection{Researcher}
\textit{Universidade Federal de Pernambuco} \hfill \textit{July 2016 - January 2019}
\textit{Recife, Pernambuco, Brazil}
\begin{itemize}
    \item Developed analog and embedded systems for biomedical therapy and monitoring applications
    \item Developed electrical stimulator for gait improvement in Parkinson's patients, using MSP430 microcontroller and real-time inertial feedback (I²C accelerometers/gyroscopes) for stimulation timing control
    \item Designed power control and switching circuit for hyperhidrosis treatment, using CMOS transistor arrays, MSP430-based PWM modulation, LTspice simulations, and PCB prototyping in Proteus
    \item Contributed to experimental validation and interface control of FES systems for neuromuscular stimulation
\end{itemize}
\textit{Skills:} System on a Chip (SoC), C, Biomedical Electronics, Electronics Design, Embedded Systems, MATLAB, Signal Processing, MSP430, Launchpad TI(Texas Instruments), Bare-Metal Embedded, Real-Time Operating Systems (RTOS), I2C

\subsection{Undergraduate Researcher – NFC Vital Sign Acquisition System}
\textit{Universidade Federal de Pernambuco} \hfill \textit{July 2017 - July 2018}
\textit{Recife, Pernambuco, Brazil}
\begin{itemize}
    \item Developed passive embedded system based on NXP/Freescale FRDM-K64F microcontroller, powered via NFC and capable of acquiring vital signs such as heart rate and blood oxygen saturation without external power source
    \item Designed and integrated NFC-based energy harvesting architecture using NTAG I²C Plus interface (Class 6), enabling biomedical signal acquisition (MAX30100) and seamless wireless communication with Android application
    \item Project presented at CBEB (Brazilian Congress of Biomedical Engineering) — largest biomedical engineering conference in Brazil — and awarded Ricardo Ferreira Award to Talented Young Scientist, recognized as best undergraduate scientific research of the year (2018)
\end{itemize}
\textit{Skills:} System on a Chip (SoC), Android Studio, Embedded Systems, NFC, NXP, Signal Processing, Radio-Frequency Identification (RFID), Bare-Metal Embedded, Biomedical Electronics, Freescale

\subsection{FPGA Engineer Intern}
\textit{CETENE} \hfill \textit{August 2015 - December 2015}
\textit{Recife, Pernambuco, Brazil}
\begin{itemize}
    \item Contributed to initial development and simulation of custom SDRAM controller IP for Altera FPGAs, using VHDL and ModelSim testbenches
    \item Contributed to development and simulation of FPGA components for low-power RFID reader project, aligned with EPCglobal Class 1 Gen 2 protocol
    \item Assisted in integration of Alien RFID Readers into plant inspection application, supporting configuration, testing, and operational validation
\end{itemize}
\textit{Skills:} Radio Frequency (RF), FPGA prototyping, VHDL, RFID Applications, Intel Quartus Prime, ModelSim, SystemVerilog

% Technical Skills
\section{Technical Skills}

\subsection{Programming Languages \& Embedded Systems}
\begin{itemize}
    \item \textbf{C/C++}: Embedded systems development, Real-time programming, FreeRTOS, ChibiOS/RT, Bare-Metal Embedded
    \item \textbf{C\#}: Industrial applications, Real-time image processing, GUI development, WPF Development
    \item \textbf{Python}: Industrial automation, Data processing, Scripting, PyTorch, Anaconda
    \item \textbf{VHDL/Verilog}: FPGA design, IP core development, Digital signal processing, FPGA prototyping
\end{itemize}

\subsection{FPGA \& Hardware Design}
\begin{itemize}
    \item \textbf{FPGA}: Xilinx UltraScale+, VHDL, IP cores, Real-time processing, Field-Programmable Gate Arrays
    \item \textbf{Microcontrollers}: STM32, MSP430, Infineon XMC, FreeRTOS, TI DSPs
    \item \textbf{PCB Design}: Altium Designer, Electronics Design, Bare-Metal Embedded
\end{itemize}

\subsection{Development Tools \& Platforms}
\begin{itemize}
    \item \textbf{Embedded Linux}: Yocto Project, Kernel, Devicetree, U-Boot, Bootloader, Device Drivers
    \item \textbf{Computer Vision}: OpenCV, Machine Learning, Data Analysis
    \item \textbf{Signal Processing}: MATLAB, Real-Time Operating Systems (RTOS)
    \item \textbf{Communication}: I2C, SPI, UART, TCP/IP, NFC, RFID Applications
\end{itemize}

\subsection{Industrial Automation \& Robotics}
\begin{itemize}
    \item \textbf{Industrial}: KUKA robots, PROFINET, IO-Link, PLC programming, Robot Framework
    \item \textbf{Protocols}: PROFIsafe, EtherCAT, Data Acquisition
    \item \textbf{SCADA}: PLC Siemens, SIMATIC STEP 7, Programmable Logic Controller (PLC)
\end{itemize}

% Key Projects
\section{Key Projects}
\begin{itemize}
    \item \textbf{NIR/SWIR Vision Systems}: Developed advanced spectral imaging systems for industrial process analysis using custom C\# software and FPGA acceleration, achieving 90\% cost reduction compared to initial project estimates
    \item \textbf{FPGA IP Core Development}: Designed and implemented convolutional kernel acceleration IP cores for Xilinx UltraScale+ devices, significantly reducing latency and hardware utilization
    \item \textbf{Robotic Cell Integration}: Engineered complete KUKA robot control systems with safety protocols and precision verification
    \item \textbf{Industrial Control Patent}: Co-authored engineering and industrial control patent integrated into a €100+ million project
\end{itemize}

% Academic Mentorship
\section{Academic Mentorship}
\begin{itemize}
    \item \textbf{Mentor:} B. Michalewicz \\
    \textit{Development of a Frame Grabber Application on a Xilinx Zynq-Based Heterogeneous Platform}
    
    \item \textbf{Mentor:} P. Matanzas \\
    \textit{Design and Implementation of an FPGA-based Acceleration System for Real-Time Image Processing}
    
    \item \textbf{Mentor:} J. Barturen \\
    \textit{Vulnerability Analysis in Industrial Automation: Pentesting in Sinumerik ONE Architectures with Profinet and IO-Link Wireless Communications}
    
    \item \textbf{Mentor:} P. Matanzas \\
    \textit{Design and Implementation of a Customized Linux Distribution for Real-Time Image Processing in Industrial Environments}
\end{itemize}

% Technical Certifications
\section{Technical Certifications}
\begin{itemize}
    \item \textbf{Verilog for an FPGA Engineer with Xilinx Vivado Design Suite}
    \item \textbf{Embedded Linux Using Yocto}
    \item \textbf{Programación TPE, Nivel B - FANUC Europe}
    \item \textbf{Certified LabVIEW Associate Developer (National Instruments)}
\end{itemize}

% Languages
\section{Languages}
\begin{itemize}
    \item \textbf{Portuguese}: Native or bilingual proficiency
    \item \textbf{Spanish}: Full professional proficiency
    \item \textbf{English}: Professional working proficiency
\end{itemize}

\end{document} 